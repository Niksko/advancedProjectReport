\section{Conclusion and future directions}

Over this semester we have succeeded in developing a basic package for performing computational evolutionary dynamics calculations in Julia.
We are able to simulate the Moran process and the replicator-mutator equation and display the results of our simulationas, as well as being able to compute some useful quantities such as stationary distributions.
We have found Julia to be a user friendly language to program in, and it has many advanced features, despite offering some resistance at times due to the fact that it is still under heavy development.

One point of note is that there is currently no docstring convention for Julia (though one is being discussed).
As such the documentation of our code is far from comprehensive, and this something that we would hope to address in the future.

Other future goals for JEDy include

\begin{itemize}
    \item Make JEDy an official Julia package by registering it at the Julia Github repository.
    \item Add additional functionality for performing computations including different simulation processes.
    \item Migrate the data visualisation functionality from PyPlot (a Julia wrapper around the mature Python library matplotlib) to a native Julia package such as Gadfly.
    \item Clean up the code to make it more readable and more modular.
    \item Allow for the use of more general games, ie. games with more than two players.
    \item Optimise the functions for memory usage and performance.
\end{itemize}
