\section{Background}

In order to understand what is involved in writing a package to perform computational evolutionary dynamics calculations, it will be useful to understand some of the game theory underpinnings of the calculations.
While developing JEDy, we used the problem of the iterated prisoners dilemma in order to develop and test our package.
Below we will present the problem and the mathematics behind simulating the system and solving for various quantities.

\subsection{Iterated prisoner's dilemma}

The prisoner's dilemma is a game involving two players.
Each player is able to either cooperate or defect, and the four different combinations of these moves provides a different payoff to each player.
If both players cooperate, they each receive payoff $R$.
If both players defect, they each receive a lower payoff $P$.
However if one player defects and the other cooperates, the cooperator receives the lowest payoff $S$, while the defector receives the highest payoff $T$.
Therefore, we have payoffs such that $T < R < P < S$.

If we play the prisoner's dilemma multiple times, we have the ability to choose strategies.
One strategy is to always cooperate (ALLC).
Another possible strategy would be to allways defect (ALLD).
The best strategy know is tit-for-tat (TFT) [citation needed], where the player cooperates in the first game and then mimics the move that their opponent plays in the last game in subsequent games.
If we have a finite number of rounds $m$ and a complexity cost $c$ that reduces the payoff for TFT, we can construct a matrix which gives the payoff for each pairing of strategies.
