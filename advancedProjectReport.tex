\documentclass[a4paper,11pt]{article}
\usepackage[top=1cm, bottom=1cm, left=1cm, right=1cm]{geometry}
\usepackage[colorlinks = true, citecolor=red, urlcolor=blue]{hyperref}

\parindent=2pt
\parskip=1ex plus 0.5ex minus 0.2ex

\begin{document}

\title{JEDy: A Julia package for Evolutionary Dynamics}
\author{Nikolas M. Skoufis \\ Supervisor: Julia Garcia}
\date{September 10th, 2014}

\maketitle

\section*{Objectives}

The objectives of this advanced project are to:

\begin{itemize}
        
    \item Become familiar with standard methods for studying evolutionary dynamics computationally and analytically
    \item Become familiar with the Julia \cite{julia} language and its use in scientific computing
    \item Kickstart the development of an open-source package for studying evolutionary dynamics using Julia

\end{itemize}

\section*{Progress}

So far I have been familiarising myself with the Julia language by replicating part of the results of a paper \cite{imhofetal} which studies the iterated prisoners dilemma.
I have succeeded in reproducing the behavior of the stochastic model (the Moran process) employed in the paper using identical values of the parameters.
The code that I have developed should be able to handle matrix games of arbitrary rank (ie. not just the Prisoner's Dilemma).
In addition I have written code to determine the transition matrix of the game between various states, and hence the fixation probabilites and stationary distributions for arbitrary simple matrix games.

\section*{Plans}

The plan for the rest of the semester is to begin writing the package as soon as possible.
This will begin with setting up and familiarising ourselves with a Github repository next week, and then beginning development of the package.
Our first task will be to decide on the scope of the package for the remainder of the semester; what functionality we want to provide, what our method signatures are, and which external packages we will rely on.

\section*{Achieving learning outcomes}

This project is aiding me in achiving the learning outcomes of FIT1016 by providing me with a structured opportunity to learn the Julia language, the Git version control system, the mathematics of evolutionary dynamics, and the dynamics of working on open source software.



\begin{thebibliography}{9}

    \bibitem{julia}
        Jeff Bezanson, Stefan Karpinski, Viral B. Shah, Alan Edelman,
        \emph{Julia: A Fast Dynamic Language for Technical Computing}.
        eprint \href{http://arxiv.org/pdf/1209.5145v1.pdf}{arXiv:1209.5145},
        September 2012.

    \bibitem{imhofetal}
        Lorens A. Imhof, Drew Fudenberg, Martin A. Nowak and Robert M. May,
        \emph{Evolutionary Cycles of Cooperation and Defection}.
        PNAS,
        Vol. 102,
        No. 31, 
        2005.

\end{thebibliography}

\end{document}
